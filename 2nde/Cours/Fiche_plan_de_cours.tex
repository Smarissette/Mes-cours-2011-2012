%Un site pour personnaliser son latex : http://www.schneeflocke.net/index.php?section=latex/index

% Tout ce qui est compris entre le caractère % et une fin de ligne est un
% commentaire ignor\'{e} par LaTeX

%iconv -f utf8 -t latin1 fichier.tex > fichier-iso.tex

% Un fichier LaTeX commence par une commande \documentclass qui d\'{e}clare le type
% de document

\documentclass[10.5pt]{article}

\usepackage[french]{babel} % Pour adopter les règles de typographie fran\c{c}aise

%\usepackage[T1]{fontenc} % Pour que LaTeX comprenne les caractères accentu\'{e}s ;
                         % norme iso-8859, cela risque de ne pas marcher avec
                         % des fichiers cr\'{e}\'{e}s sous windows
\usepackage[utf8]{inputenc}
\usepackage{multicol}
\usepackage{fullpage}
\usepackage{color}
\usepackage{amsmath} % Les biblioth\`{e}ques LaTeX de l'American
                     % Mathematical Society sont pleines de macros
                     % int\'{e}ressantes (voire indispensables).
\usepackage{graphics}
\usepackage[pdftex]{graphicx} 
\usepackage[]{graphicx}
\definecolor{perso}{RGB}{30,30,160}

\newcommand{\promille}{\def\pourmille{\hbox{$\,^0\!/_{00}$}}}

\definecolor{jauneboite}{RGB}{244,232,168}
\makeatletter\newenvironment{boite}{%
   \begin{lrbox}{\@tempboxa}\begin{minipage}{0.9 \linewidth}}{\end{minipage}\end{lrbox}%
   \colorbox{jauneboite}{\usebox{\@tempboxa}}
}\makeatother


%\newenvironment{boite}{%
%\colorbox[RGB]{244,232,170}{
%\begin{minipage}{\linewidth}
%}%
%{%
%\end{minipage}}
%}


\begin{document}

%\addtolength{\textheight}{50pt}
%\newenvironment{changemargin}[2]{\begin{list}{}{
%\setlength{\leftmargin}{0pt}
%\setlength{\rightmargin}{0pt}
%\addtolength{\leftmargin}{#1}
%\addtolength{\rightmargin}{#2}
%}\item }{\end{list}}

%\begin{changemargin}{-1.8 cm}{-1.8 cm}

%\begin{multicols}{2}
%\end{multicols}

\begin{center}
\textsc{Thème 1 -- La Terre dans l'Univers, la vie et l'évolution du vivant : une planète habitée} \\
~~ \\
\Large{\textbf{Chapitre 1 -- Les conditions de la vie :}} \\
\Large{\textbf{une particularité de la Terre ?}}
\end{center}

\begin{center}
\begin{tabular}{|*{2}{l|}c|}
\hline
Plan du chapitre & Je dois être capable de : & Autoévaluation \\ \hline
& & \\
\textbf{\large{I- La Terre dans le système solaire}} &  &  \\
& $\oplus$ donner la définition de : & :-)  \quad \quad  :-/ \quad \quad  :-(  \\
1- L'organisation du système solaire & système solaire, étoile, planète rocheuse, & \\
& planète gazeuse, astéroïde, comète & \\
2- Les différents objets du système solaire & & \\
& $\oplus$ schématiser le système solaire & :-)  \quad \quad  :-/ \quad \quad  :-(  \\
& à l'échelle & \\
& & \\
& & \\ \hline 
& & \\
\textbf{\large{II- La Terre, une planète habitable}} & & \\ 
\textbf{\large{et habitée}} & & \\
& $\oplus$ citer les caractéristiques de la Terre & :-)  \quad \quad  :-/ \quad \quad  :-(  \\
1- Les conditions de la Terre sont favorables &  qui permettent la présence de la vie &  \\
à la vie & & \\
& $\oplus$ les relier à sa place dans le système & :-)  \quad \quad  :-/ \quad \quad  :-(  \\ 
2- D'autres planètes habitées ? & solaire & \\
& & \\
& $\oplus$ discuter des arguments d'une possible & :-)  \quad \quad  :-/ \quad \quad  :-(  \\ 
& vie sur d'autres planètes & \\
& & \\ \hline
\end{tabular}
\end{center}

%ligne courante du texte
%\begin{minipage}[t]{4cm}
%:-)  :-/  :-( 
%\end{minipage}

%$\ddot\smile$

%\end{changemargin}
\end{document}
